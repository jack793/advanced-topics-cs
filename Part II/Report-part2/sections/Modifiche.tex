\documentclass[../Relazione.tex]{subfiles}

\begin{document}
\section{Modifiche a Ticket Management}
    Rispetto alla rete di Petri riferita alla prima parte, è stata apportata una sola modifica.
    Abbiamo deciso di attribuire alla transizione \texttt{Check Completeness} la priorità \textit{P\_HIGH}. Eseguendo più volte la simulazione per trovare il corretto numero di impiegati da assumere, ci siamo accorti che era presente un caso limite: al 182-esimo giorno sarebbe stato prodotto un token in \texttt{Waiting Tickets} e questo avrebbe fatto si che la transizione \texttt{Prepare for CC} avesse la stessa probabilità di effettuare il fire di \texttt{Check Completeness}. Questo non sarebbe stato corretto poiché anche al giorno 182 è possibile chiudere il ticket.
    
    Grazie alla priorità conferita siamo così sicuri che, nel caso in cui ci sia la possibilità che entrambe queste transizioni possano effettuare il fire lo stesso giorno (il 182-esimo), la scelta ricada sempre su \texttt{Check Completeness}.
        
         
\end{document} 
		